%%%%%%%%%%%%%%%%%%%%%%%%%%%%%%%%%%%%%%%%%%%%%%%%%%

% Standard packages

\usepackage{amsmath}

\usepackage{amssymb}

%\usepackage{calculator}

\usepackage{chemfig}

\usepackage{enumitem}

\usepackage{fancyhdr}

\usepackage[top=3.14cm, bottom=3cm, outer=2.5cm, inner=2.5cm]{geometry}

\usepackage{graphicx}

\usepackage[hidelinks]{hyperref}

\usepackage{lastpage}

\usepackage{mathtools}

\usepackage{mhchem}

\usepackage{multicol}

%\usepackage{multirow}

\usepackage{pgfplots}

%\usepackage{pgf-pie}

\usepackage{tikz}

%\usepackage{titling}

\usepackage{lipsum}

\usepackage[bottom]{footmisc}

\usepackage{xcolor,colortbl}

\usepackage{mdframed}

\usepackage[backend=bibtex]{biblatex}

\usepackage{csquotes}

\usepackage{anyfontsize}

%%%%%%%%%%%%%%%%%%%%%%%%%%%%%%%%%%%%%%%%%%%%%%%%%%

% Package settings

\pgfplotsset{compat=1.18}

%%%%%%%%%%%%%%%%%%%%%%%%%%%%%%%%%%%%%%%%%%%%%%%%%%

% Document settings

\pagestyle{fancy}

\fancyhead{}	
\fancyfoot{}

\fancyhead[l]{\title{}}
\fancyhead[r]{\author{}}

\fancyfoot[l]{\school}
\fancyfoot[r]{Side \thepage{}\hspace{1mm}af~\pageref{LastPage}}

\renewcommand{\contentsname}{Indhold}

\renewcommand{\maketitle}{
	\begin{center}        
        {\fontsize{32pt}{32pt}\selectfont \title{}}

        \vspace{0.25cm}

        {\fontsize{16pt}{16pt}\selectfont \author{}, \subject{}}

        \vspace{0.15cm}

        {\fontsize{12pt}{12pt}\selectfont \textcolor{black!65}{\date{}} }
    \end{center}
}

\newcommand{\articlebreak}{
	\\

	\noindent
}

\def\RED{\gdef\printatom##1{\color{red}\ensuremath{\mathrm{##1}}}}
\def\BLUE{\gdef\printatom##1{\color{blue}\ensuremath{\mathrm{##1}}}}
\def\BLACK{\gdef\printatom##1{\color{black}\ensuremath{\mathrm{##1}}}}

\setlength{\fboxsep}{0pt}
\setlength{\fboxrule}{1px}

\AtBeginEnvironment{align}{\setcounter{equation}{0}}

%%%%%%%%%%%%%%%%%%%%%%%%%%%%%%%%%%%%%%%%%%%%%%%%%%

% Commands

\newcommand*\Eval[3]{\bigg[#1\bigg]_{#2}^{#3}}

\DeclareMathOperator{\di}{d\!}

\newcommand*\dydx{\frac{\di y}{\di x}}

\DeclareMathOperator{\txtdeg}{\text{\textdegree{}}}

\DeclareMathOperator{\for}{ \ \text{ for } \ }

\newcommand*\hm[1]{$\displaystyle #1$}

\newcommand*\chs[2]{\section*{\quad {#1}) \indent #2}}
\newcommand*\chss[2]{\subsection*{\quad {#1}) \indent #2}}
\newcommand*\chsss[2]{\subsubsection*{\quad {#1}) \indent #2}}

\newcommand*\fx{f_{(x)}}

\newcommand*\img[2]{
	\begin{center}
		\includegraphics[width=#1\textwidth]{src/#2}
	\end{center}
}

\newcommand*\pu[2]{$#1 \, (#2)$}

\newcommand*\dx{\di x}
\newcommand*\dy{\di y}

\newcommand*\diff{\frac{\delta x}{\delta y}}
\newcommand*\diffx{\delta x}
\newcommand*\diffy{\delta y}

\newcommand{\correct}{\cellcolor[HTML]{08f26e}}
\newcommand{\xcorrect}{\cellcolor[HTML]{90CE90}}
\newcommand{\wrong}{\cellcolor[HTML]{ed4337}}
\newcommand{\xwrong}{\cellcolor[HTML]{E57F7E}}
\newcommand{\gray}{\cellcolor[HTML]{bbbbbb}}

%%%%%%%%%%%%%%%%%%%%%%%%%%%%%%%%%%%%%%%%%%%%%%%%%%